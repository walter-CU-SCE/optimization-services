\documentclass[11pt]{article}
\usepackage{graphics,graphicx}
%\usepackage[dvips]{graphics,graphicx}
\DeclareGraphicsExtensions{.ps,.jpg,.eps,.pdf,.png}
\usepackage{boxedminipage,amsmath,amsfonts}
\usepackage{url}
%\usepackage{secdot}
%\usepackage{natbib}
\usepackage{verbatim}
%\usepackage{moreverb}
\usepackage{enumerate}
\usepackage{makeidx}
\bibliographystyle{plain}
%\makeindex


% Define the hangref environment used for the References list:
\newenvironment{hangref}
  {\begin{list}{}{\setlength{\itemsep}{4pt}
  \setlength{\parsep}{0pt}\setlength{\leftmargin}{+\parindent}
  \setlength{\itemindent}{-\parindent}}}{\end{list}}

% Set the page margins to 1 inch all around:
\marginparwidth 0pt\marginparsep 0pt \topskip 0pt\headsep
0pt\headheight 0pt \oddsidemargin 0pt\evensidemargin 0pt
\textwidth 6.5in \topmargin 0pt\textheight 9.0in
\newtheorem{theorem}{Theorem}


\begin{document}

\title{Generic parser implementation }
\vskip 2in
\author{Horand Gassmann, Jun Ma,  Kipp Martin}
\maketitle

\begin{abstract}
This documents makes some recommendations concerning elements, rules, names, content and structure
of bison rule files for parsing documents used within the OS framework. The emphasis is on uniformity;
computational efficiency is secondary; it is expected. The guidelines were developed in parallel with the
development of the OSrL parser, but they have application for other parsers. 
\end{abstract}

\begin{enumerate}

\item    
For ease of development, trouble shooting and maintenance it is useful to have treatment of the 
different elements that is as uniform as possible. Computational efficiency is secondary; it is expected 
that the compiler will be able to deal with such issues. 

\item \label{enum:element}
	Every $<${\it element}$>$ is parsed using three production rules: {\it  elementStart\/}, {\it elementAttributes\/} 
and {\it elementContent\/}. 

\item
	{\it elementStart\/} matches the opening tag (�$<${\it element\/}�); its code section can be used 
to verify that the element was indeed expected at this spot, particularly in cases where this is hard 
to infer from the Schema. There are two instances when such checks need to be made:

\begin{enumerate}
\item	If the elements do not have to appear in any particular order, 
it is necessary to verify that there was no prior occurrence of this $<${\it element}$>$ within the scope 
of its parent.
\item If the element is contained in a repeat group, we must make sure that there are not 
more occurrences than specified in the {\it numberOf}$\ldots$ attribute of its parent.
\end{enumerate}
In addition the code can be used to initial the attribute list. The occurrence of attributes is tracked 
with indicators {\it xxxAttPresent}, which can be set to {\tt false} in this section. If an element 
has an optional {\it numberOf}$\ldots$ attribute, the variable holding the number of these items 
should be set to zero here to provide a default when the {\it numberOf}$\ldots$ attribute is missing.

\item	{\it elementAttributes\/} is included as a separate rule so that checks can be made after 
the entire list of attributes has been processed. It is necessary to check that all mandatory attributes 
have indeed been provided, and there may be other checks as needed. The production rule is

{\it elementAttributes: elementAttList}

where {\it elementAttList\/} is a standard list rule, which expands into

{\it elementAttList:} $\vert$ {\it elementAttList elementAtt\/}.

\item	{\it elementAtt\/} matches any of a list of attributes allowed under the current element, as in

{\it elementAtt: elementxxxAtt\/} $\vert$ {\it elementyyyAtt} $\vert$ {\it elementzzzAtt} $\ldots$

\item	Each {\it elementxxxAtt\/} is used to perform specific data checks, such as 
membership in an enumerative list, nonnegativity, etc., and to store the attribute value into 
the internal data structure. Moreover, attribute names, unlike element names, tend to be reused frequently. 
Thus {\it elementxxxAtt\/} may be a generic rule shared among many elements. 

\item	{\it elementxxxAtt\/} is also used to verify that the attribute has not been seen previously 
within the scope of the current element, to change its status from not present to present, 
and to assign the attribute value to a temporary variable.

\item	If an element allows only a single attribute, the above can be streamlined, 
the rule {\it elementxxxAtt\/} replacing the rule {\it elementAttributes\/}.

\item	If an element has no attributes, this rule is simply omitted.

\item	An element attribute may be used to record the number of child elements that are given in an array list. 
The parser records the number of child elements actually encountered and compares against the declared number. 
Any discrepancy is recorded. Such {\it numberOf}$\ldots$ attributes also allocate the storage space 
for the child elements and set the counter to~0. 

\item	{\it elementContent\/} can be empty or nonempty. This is normally expressed by the rule

{\it elementContent: elementEmpty} $\vert$ {\it elementBody}

In some rare cases modifications from this rule are needed in order to avoid reduce/reduce conflicts 
when an element has several optional children that must occur in a particular sequence, for instance

{\it variables: variableValues variableValuesString basisStatus otherVariableResultsArray}

where each of the child elements may be omitted --- or indeed all of them together.

\item	The code section in the {\it elementContent\/} rule can be used for consistency checks, 
storage of information into the data structure and, most importantly, to increment counters. 

\item	Empty element content is typically either ``$><$/{\it element\/}$>$'' or simply ``/$>$''. 
Code may be needed to detect empty element content and throw an appropriate error.

\item	{\it elementBody\/} expands into a variety of different patterns, as needed. There could be
\begin{enumerate}
\item	an array of $<${\it child\/}$>$ elements, which is distinguished from an element list 
by using the rule name {\it childArray\/}, which expands into 

{\it childArray:} $\vert$ {\it childArray childElement}
\item	several children in arbitrary order ({\it childList}) with a similar expansion
\item	other constructs as appropriate.
\end{enumerate}
\item	Each $<${\it childElement\/}$>$ would then be treated again as under point~\ref{enum:element}.

\end{enumerate}

\end{document}