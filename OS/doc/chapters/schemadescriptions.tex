
\section{OS Protocols}\label{section:schemadescriptions}

The objective of OS is to provide a set of standards for representing optimization instances, results, solver options,
and communication between clients and solvers in a distributed environment using Web Services.  These standards are
specified by W3C XSD schemas. The schemas for the OS project are contained in the {\tt schemas} folder under the
{\tt OS} root. There are numerous schemas in this directory that are part of the OS standard.
For a full description of all the schemas see  Ma \cite{junma2005}.  We briefly discuss the standards most relevant
to the current version of the OS project.


\subsection{OSiL (Optimization Services instance Language)} \label{section:osilschema}
OSiL\index{OSiL|(} is
an XML-based language for representing instances of large-scale
optimization problems including linear programs, mixed-integer programs,
quadratic programs, and very general nonlinear programs.

OSiL stores optimization problem instances as XML files.  Consider the following problem instance, which is a
modification of an example of Rosenbrock\index{Rosenbrock, H.H.@{\it Rosenbrock, H.H.}}~\cite{rosenbrock1960}:
%
\begin{alignat}{2}
& \mbox{Minimize} & \quad (1 - x_{0})^{2} + 100(x_{1} - x_{0}^{2})^{2} + 9x_{1} \label{eq:roobj}\\
& \mbox{s.t.} & \quad x_{0} + 10.5 x_{0}^{2} + 11.7 x_{1}^{2} + 3x_{0}x_{1}  &\le 25  \label{eq:ro1}\\
& & \ln(x_{0} x_{1}) + 7.5 x_{0} + 5.25 x_{1} &\ge 10 \label{eq:ro2}\\
& & x_{0}, x_{1} &\ge 0 \label{eq:ro3}
\end{alignat}


There are two continuous variables, $x_{0}$ and $x_{1}$, in this instance, each with a lower bound of 0.
Figure~\ref{figure:variableselement} shows how we represent this information in an XML-based OSiL file.
Like all XML files, this is a text file that contains both {\it markup} and {\it data}. In this case there
are two types of markup, {\it elements} (or {\it tags}\/) and {\it attributes} that describe the elements.
Specifically, there are a {\tt <variables>} element and two {\tt <var>} elements. Each {\tt <var>}
element has attributes {\tt lb}, {\tt name}, and {\tt type} that
describe properties of a decision variable: its lower bound, ``name'', and
domain type (continuous, binary, general integer).


\begin{figure}[b]
\centering
   \small {\obeyspaces\let =\
\fbox{\tt\begin{tabular}{@{}l@{}}
<variables numberOfVariables="2">\\[\Sb]
    <var lb="0" name="x0" type="C"/>\\[\Sb]
    <var lb="0" name="x1" type="C"/>\\[\Sb]
</variables>\\[\Sb]
\end{tabular} }} \medskip
\caption{The {\tt <variables>} element for the example (1)--(4).}\label{figure:variableselement}
\end{figure}


     To be useful for communication between solvers and modeling
languages, OSiL instance files must conform to a standard.
An XML-based representation standard is imposed
through the use of a {\em W3C XML Schema.} The W3C, or World Wide
Web Consortium (\url{www.w3.org}), promotes standards for
the evolution of the web and for interoperability between web
products.  XML Schema (\url{www.w3.org/XML/Schema}) is one
such standard.  A schema specifies the elements and attributes that
define a specific XML vocabulary. The W3C XML Schema is thus a schema
for schemas; it specifies the elements and attributes for a schema
that in turn specifies elements and attributes for an XML
vocabulary such as OSiL. An XML file that conforms to a
schema is called {\it valid} for that schema.

     By analogy to object-oriented programming, a schema is akin to a header file in C++ that defines the members and methods in a class.  Just as a class in C++ very explicitly describes member and method names and properties, a
schema explicitly describes element and attribute names and properties.

{\small
\begin{figure}[b]
   \small {\obeyspaces\let =\
\makebox[0in][t]{\fbox{\tt\begin{tabular}{@{}l@{}}
<xs:complexType name="Variables">\\[\Sb]
    <xs:sequence>\\[\Sb]
        <xs:element name="var" type="Variable" maxOccurs="unbounded"/>\\[\Sb]
    </xs:sequence>\\[\Sb]
    <xs:attribute name="numberOfVariables"\\[\Sb]
            type="xs:positiveInteger" use="required"/>\\[\Sb]
</xs:complexType>\\[\Sb]
\end{tabular} }}} \medskip
\caption{The {\tt  Variables} complexType  in the OSiL
schema.}\label{figure:osilvariables}
\end{figure}
}%end small


{\small
\begin{figure}[b]
   \small {\obeyspaces\let =\
\makebox[0in][t]{\fbox{\tt\begin{tabular}{@{}l@{}}
<xs:complexType name="Variable">\\[\Sb]
    <xs:attribute name="name" type="xs:string" use="optional"/>\\[\Sb]
    <xs:attribute name="init" type="xs:string" use="optional"/>\\[\Sb]
    <xs:attribute name="type" use="optional" default="C">\\[\Sb]
        <xs:simpleType>\\[\Sb]
            <xs:restriction base="xs:string">\\[\Sb]
                <xs:enumeration value="C"/>\\[\Sb]
                <xs:enumeration value="B"/>\\[\Sb]
                <xs:enumeration value="I"/>\\[\Sb]
                <xs:enumeration value="S"/>\\[\Sb]
            </xs:restriction>\\[\Sb]
        </xs:simpleType>\\[\Sb]
    </xs:attribute>\\[\Sb]
    <xs:attribute name="lb" type="xs:double" use="optional" default="0"/>\\[\Sb]
    <xs:attribute name="ub" type="xs:double" use="optional" default="INF"/>\\[\Sb]
</xs:complexType>\\[\Sb]
\end{tabular} }}} \medskip
\caption{The {\tt  Variable} complexType in the OSiL
schema.}\label{figure:osilvar}
\end{figure}
} %end small



Figure~\ref{figure:osilvariables} is a piece of our schema for OSiL. In W3C XML Schema jargon, it defines a {\it complexType,}  whose purpose is to specify elements and attributes that are allowed to appear in a valid XML instance file such as the one excerpted in Figure~\ref{figure:variableselement}. In particular, Figure~\ref{figure:osilvariables} defines the complexType named {\tt Variables}, which
comprises an element named {\tt <var>} and an attribute named {\tt
numberOfVariables}. The {\tt numberOfVariables} attribute is of a
standard type {\tt positiveInteger}, whereas the {\tt <var>} element is
a user-defined complexType named {\tt Variable}. Thus the complexType {\tt
Variables} contains a sequence of {\tt <var>} elements that
are of complexType {\tt Variable}. OSiL's schema must also provide a
specification for the {\tt Variable} complexType, which is shown in
Figure~\ref{figure:osilvar}.

In OSiL the linear part of the problem is stored in the  {\tt
<linearConstraintCoefficients>} element, which stores the coefficient
matrix using three arrays as proposed in the earlier LPFML schema
\cite{fourer2005a}.  There is a child element of {\tt <linearConstraintCoefficients>} 
to represent each array: {\tt <value>} for an array of nonzero coefficients, 
{\tt <rowIdx>} or {\tt <colIdx>} for a corresponding array of row indices or column indices, 
and {\tt <start>} for an array that indicates where each row or column begins in the previous two arrays.
This is shown in Figure~\ref{figure:rowlistMatrix}.


\begin{figure}[ht]
\centering
   \small {\obeyspaces\let =\
\fbox{\tt\begin{tabular}{@{}l@{}}
<linearConstraintCoefficients numberOfValues="3">\\[\Sb]
    <start>\\[\Sb]
        <el>0</el><el>2</el><el>3</el>\\[\Sb]
    </start>\\[\Sb]
    <rowIdx>\\[\Sb]
        <el>0</el><el>1</el><el>1</el>\\[\Sb]
    </rowIdx>\\[\Sb]
    <value>\\[\Sb]
        <el>1.</el><el>7.5</el><el>5.25</el>\\[\Sb]
    </value>\\[\Sb]
</linearConstraintCoefficients>\\[\Sb]
\end{tabular} }} \medskip\\[\Sb]
\caption{The {\tt <linearConstraintCoefficients>} element for constraints
(\ref{eq:ro1}) and (\ref{eq:ro2}).}\label{figure:rowlistMatrix}
\end{figure}

The quadratic part of the problem is represented  in Figure~\ref{figure:qterms}.

\begin{figure}[ht]
\centering
   \small {\obeyspaces\let =\
\fbox{\tt\begin{tabular}{@{}l@{}}
<quadraticCoefficients numberOfQuadraticTerms="3">\\[\Sb]
     <qTerm idx="0" idxOne="0" idxTwo="0" coef="10.5"/>\\[\Sb]
     <qTerm idx="0" idxOne="1" idxTwo="1" coef="11.7"/>\\[\Sb]
     <qTerm idx="0" idxOne="0" idxTwo="1" coef="3."/>\\[\Sb]
</quadraticCoefficients>\\[\Sb]
\end{tabular} }} \medskip
\caption{The {\tt <quadraticCoefficients>} element for constraint (\ref{eq:ro1}).}
\label{figure:qterms}
\end{figure}

The nonlinear part of the problem is given in Figure~\ref{figure:roobjnlnode}.



{\small
\begin{figure}[t]
\centering
   \small {\obeyspaces\let =\
\fbox{\tt\begin{tabular}{@{}l@{}}
<nl idx="-1">\\[\Sb]
     <plus>\\[\Sb]
          <power>\\[\Sb]
               <minus>\\[\Sb]
                    <number value="1.0"/>\\[\Sb]
                    <variable coef="1.0" idx="0"/>\\[\Sb]
               </minus>\\[\Sb]
               <number value="2.0"/>\\[\Sb]
          </power>\\[\Sb]
          <times>\\[\Sb]
               <power>\\[\Sb]
                    <minus>\\[\Sb]
                         <variable coef="1.0" idx="0"/>\\[\Sb]
                         <power>\\[\Sb]
                              <variable coef="1.0" idx="1"/>\\[\Sb]
                              <number value="2.0"/>\\[\Sb]
                         </power>\\[\Sb]
                    </minus>\\[\Sb]
                    <number value="2.0"/>\\[\Sb]
               </power>\\[\Sb]
               <number value="100"/>\\[\Sb]
          </times>\\[\Sb]
     </plus>\\[\Sb]
</nl>\\[\Sb]
\end{tabular} }} \medskip\\[\Sb]
\caption{The {\tt <nl>} element for the nonlinear part of the objective (\ref{eq:roobj}).}\label{figure:roobjnlnode}
\end{figure}
}

The complete OSiL representation can be found in the Appendix (Section~\ref{section:rosenbrockXML}).%
\index{OSiL|)}


\subsection{OSrL (Optimization Services result Language)} \label{section:osrlschema}
OSrL\index{OSrL|(} is an XML-based language for representing the solution of large-scale
optimization problems including linear programs, mixed-integer programs,
quadratic programs, and very general nonlinear programs.  An example solution (for the problem given in
 (\ref{eq:roobj})--(\ref{eq:ro3}) ) in OSrL format is given below.

{\small
\begin{verbatim}
<?xml version="1.0" encoding="UTF-8"?>
<?xml-stylesheet type = "text/xsl"
  href = "/Users/kmartin/Documents/files/code/cpp/OScpp/COIN-OSX/OS/stylesheets/OSrL.xslt"?>
<osrl xmlns="os.optimizationservices.org"
      xmlns:xsi="http://www.w3.org/2001/XMLSchema-instance"
      xsi:schemaLocation="os.optimizationservices.org
      http://www.optimizationservices.org/schemas/2.0/OSiL.xsd">
    <general>
        <generalStatus type="normal"/>
        <serviceName>Solved using a LINDO service</serviceName>
        <instanceName>Modified Rosenbrock</instanceName>
    </general>
    <optimization numberOfSolutions="1" numberOfVariables="2" numberOfConstraints="2"
        numberOfObjectives="1">
        <solution targetObjectiveIdx="-1">
            <status type="optimal"/>
            <variables>
                <values numberOfVar="2">
                    <var idx="0">0.87243</var>
                    <var idx="1">0.741417</var>
                </values>
                <other numberOfVar="2" name="reduced costs" description="the variable reduced costs">
                    <var idx="0">-4.06909e-08</var>
                    <var idx="1">0</var>
                </other>
            </variables>
            <objectives>
                <values numberOfObj="1">
                    <obj idx="-1">6.7279</obj>
                </values>
            </objectives>
            <constraints>
                <dualValues numberOfCon="2">
                    <con idx="0">0</con>
                    <con idx="1">0.766294</con>
                </dualValues>
            </constraints>
        </solution>
    </optimization>
\end{verbatim}
}
% Hide this stuff for now...
% The OSrL schema is also used to return timer and system statistics that are sometimes 
% gathered by the solvers themselves or generated as a result of using the {\tt knock} 
% method. (See the example given in Section~\ref{section:knock}.)
\index{OSrL|)}



\subsection{OSoL (Optimization Services option Language)} \label{section:osolschema}
OSoL\index{OSoL|(} is
an XML-based language for representing options that get passed to an optimization solver or a hosted optimization
solver Web service. It contains both standard options for generic services and extendable option tags for
solver-specific directives.
Several examples of files in OSoL format are presented in Section~\ref{section:servicemethods}.%
\index{OSoL|)}

\subsection{OSnL (Optimization Services nonlinear Language)} \label{section:osnlschema}
The OSnL\index{OSnL|(} schema is imported by the OSiL\index{OSiL} schema and is used 
to represent the nonlinear part of an optimization instance. 
This is explained in greater detail in Section~\ref{section:osexpressiontreeclass}. Also refer to
Figure~\ref{figure:roobjnlnode} for an illustration of elements from the OSnL standard. This figure represents
the nonlinear part of the objective in equation~(\ref{eq:roobj}), that is,
%
$$
(1-x_0)^2 + 100 (x_1-x_0^2)^2.
$$
\index{OSnL|)}

\subsection{OSpL (Optimization Services process Language)} \label{section:osplschema}
\index{OSpL|(}This is a standard used to enquire about dynamic process information that 
is kept by the Optimization Services registry. The string passed to the {\tt knock} 
method is in the OSpL format. See the example given in Section~\ref{section:knock}.\index{OSpL|)}

