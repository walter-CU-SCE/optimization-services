
\section{The Optimization Services (OS) Project}

The objective of Optimization Services (OS) is to provide a general framework consisting of a set of standards
for representing optimization instances, results, solver options, and communication between clients and solvers
in a distributed environment using Web Services. This COIN-OR project provides source code for libraries and
executable programs that implement OS standards.  See the COIN-OR Trac page {\tt\UrlTrac}\index{Trac system}
or the Optimization Services Home Page {\tt\UrlOs}\index{Optimization Services} for more information.

Like other COIN-OR projects, OS has a versioning system that ensures end users some degree of stability 
and a stable upgrade path as project development continues. The current stable version of OS is \OSstable, 
and the current stable release is \OSrelease\index{OS project!stable release}, based on trunk version~\OStrunk.

\ifruncode
This document provides descriptions for the following components of the OS project:
\else
The OS project provides the following:
\fi

\begin{enumerate}
\item{}  A set of XML\index{XML} based standards for representing optimization instances (OSiL)\index{OSiL}, 
optimization results (OSrL)\index{OSrL}, and optimization solver options (OSoL)\index{OSoL}. 
There are other standards, but these are the main ones. 
The schemas for these standards are described in Section~\ref{section:schemadescriptions}.

\ifruncode\else
\item{}  Open source libraries  that support and implement many of the standards.

\item{}  A robust solver and modeling language interface (API) for linear and nonlinear optimization problems.
Corresponding to the OSiL problem instance representation there is an in-memory object,
{\tt OSInstance}\index{OSInstance@{\tt OSInstance}},
along with a collection of  {\tt get()},   {\tt set()}, and {\tt calculate()} methods for accessing and creating
problem instances. This is a very general API for linear, integer, and nonlinear programs.
Extensions for other major types of optimization problems are also in the works. Any modeling language that can
produce OSiL can easily communicate with any solver that uses the OSInstance API.   
The {\tt OSInstance}\index{OSInstance@{\tt OSInstance}} object
is described in more detail in Section~\ref{section:osinstanceAPI}. The nonlinear part of the API is based on the
COIN-OR project CppAD\index{COIN-OR projects!CppAD@{\tt CppAD}} by Brad Bell ({\tt\UrlCppad}) but is written 
in a very general manner and could be used with other algorithmic differentiation packages. More detail on 
algorithmic differentiation is provided in Section~\ref{section:ad}.
\fi

\item{}  A  command line executable {\tt OSSolverService}\index{OSSolverService@{\tt OSSolverService}}  for reading
problem instances (OSiL format\index{OSiL}, AMPL  nl format\index{AMPL nl format},  
MPS format\index{MPS format}) and calling a solver either locally or on a remote server.
This is described in Section~\ref{section:ossolverservice}.

\ifruncode\else
\item{} Utilities that convert AMPL nl files  and MPS files into the OSiL XML format.
This is described in Section~\ref{section:osmodelinterfaces}.
\fi

\item{}  Standards that facilitate the communication between clients and optimization solvers using Web Services.
\ifruncode\else
In  Section~\ref{section:osagent} we describe the {\tt OSAgent}\index{OSAgent@{\tt OSAgent}} part of the OS library
that is used to create Web Services SOAP\index{SOAP protocol} packages with OSiL instances and contact a server for 
solution.
\fi

\item{}  An executable program {\tt OSAmplClient}\index{OSAmplClient@{\tt OSAmplClient}} that is designed to work with 
the AMPL\index{AMPL} modeling language. The {\tt OSAmplClient} appears as a ``solver'' to AMPL and, based on options 
given in AMPL, contacts solvers either remotely or locally to solve instances created in AMPL. This is described in
Section~\ref{section:amplclient}.

\ifruncode\else
\item{}  Server software that works with Apache Tomcat\index{Apache Tomcat} and Apache Axis\index{Apache Axis}.
This software uses Web Services technology and acts as middleware between the client that creates the instance
and the  solver on the server that optimizes the instance and returns the result. This is illustrated in
Section~\ref{section:tomcat}.

\item{}  A lightweight version of the project, {\tt OSCommon},\index{OSCommon@{\tt OSCommon}} for modeling language and 
solver developers who want to use OS API, readers and writers, without the overhead of other COIN-OR projects or any 
third-party software. For information on how to download {\tt OSCommon} see Section~\ref{section:oslite}.
\fi
\end{enumerate}

