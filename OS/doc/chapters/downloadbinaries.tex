\section{Downloading the \ifdevelop OS\else CoinAll\fi  Binaries}\label{section:obtainingbinaries}

\ifdevelop
The OS project is an open-source project  with source code under the Eclipse Public License~(EPL)%
\index{Eclipse Public License (EPL)}.
See~{\tt\UrlEpl}.  This project was initially created by Robert Fourer, Jun Ma, and Kipp Martin.
The code has been written primarily by  Horand Gassmann,   Jun Ma,  and Kipp Martin.    
Horand Gassmann,  Jun Ma,  and Kipp Martin are the COIN-OR project leaders and active developers for the OS project.
\else
The CoinAll project is actually a meta-project consisting of most of the COIN-OR solvers and supporting utility projects.  We describe how to download this project. 
\fi

%Below we describe different methods for obtaining the binaries and C++ source code.
Most users will only be interested in obtaining the binaries, which we describe  next.
%in Section~\ref{section:obtainingbinaries}. The remaining sections of this chapter deal with obtaining %the source code for the project, which will be of interest mostly to developers.
It is also possible to obtain the source code for the project, which will be of interest mostly to developers. 
\ifdevelop
Details can be found in  Section~\ref{section:downloadsource}.
\else
If binaries are not provided for a particular operating system, it may be possible to build them from the source.
For details it is best to start reading the OS web page at~{\tt\UrlOsWiki}.
\fi



%If the user does not wish to compile source code, the OS library, OSSolverService executable
%and Tomcat server software configuration are available in binary format for some operating systems.     
The repositoryof the binaries is at {\tt\UrlOsBinaries}\index{Downloading!binaries}.
%
\ifdevelop
 Unlike the source code described in Section~\ref{section:downloadwithsvn}, the binary files 
are not subject to version control and can be downloaded using an ordinary browser. 
%If binaries are not provided for a particular operating system,
%it may be possible to build them from the source code. Since the source is under version control, 
%this requires svn. (See Sections \ref{section:svn}, \ref{section:downloadwithsvn} and~\ref{section:build}.)
\fi

The binary distribution for the OS library and executables follows the following naming convention:


\begin{verbatim}
OS-version_number-platform-compiler-build_options.tgz (zip)
\end{verbatim}
For example, OS  Release 2.1.0 compiled with the Intel 9.1 compiler on an Intel 32-bit Linux system is:
\begin{verbatim}
OS-2.1.0-linux-x86-icc9.1.tgz
\end{verbatim}

For more detail on the naming convention and examples see:

\medskip
\noindent{\tt\UrlCoinNames}
\medskip

After unpacking the {\tt tgz} or {\tt zip} archives, the following folders are available.
\begin{itemize}

\item[] {\bf bin --} this directory has the executables {\tt OSSolverService}\index{OSSolverService@{\tt OSSolverService}} 
and {\tt OSAmplClient}\index{OSAmplClient@{\tt OSAmplClient}}.

\item[]  {\bf include --} the header files that are necessary in order to link against the OS library.

\item[] {\bf lib --} the libraries that are necessary for creating applications that use the OS library.

\item[] {\bf  share --} license and author information for all the projects used by the OS project.
\end{itemize}



Files are also provided for an Apache Tomcat\index{Apache Tomcat} Web server along with the associated Web service
that can
read SOAP \index{SOAP protocol} envelopes with model instances in OSiL\index{OSiL} format and/or options in 
OSoL\index{OSoL} format, call the {\tt OSSolverService}\index{OSSolverService@{\tt OSSolverService}},
and return the optimization result in OSrL\index{OSrL} format.
The naming convention\index{file naming conventions} for the server binary is
\begin{verbatim}
OS-server-version_number.tgz (.zip)
\end{verbatim}
For example, the files associated with  OS server release 2.0.0 are in the binary distribution
\begin{verbatim}
OS-server-2.0.0.tgz
\end{verbatim}
There is no platform information given since the server and related binaries were written in Java\index{Java}.
The details and use of this distribution are described in Section~\ref{section:tomcat}.



Finally for Windows users we provide Visual Studio \index{Microsoft Visual Studio} project files 
(and supporting libraries and header files) for building projects based on the OS library and libraries 
used by the OS project. The binary for this is named
\begin{verbatim}
OS-version_number-VisualStudio.zip
\end{verbatim}
For example, the necessary files associated with  OS  stable\index{OS project!stable release} 2.4 
are in the binary distribution
\begin{verbatim}
OS-2.4-VisualStudio.zip
\end{verbatim}
The binaries provided are based on Visual Studio Express 2008.  See Section \ref{section:vsexamples} for more detail.
