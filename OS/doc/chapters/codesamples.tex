\section{Code samples to illustrate the OS Project}\label{section:examples}

\ifdevelop
These example executable files are not built by running {\tt configure} and {\tt make}. 
Windows users are advised to download a binary distribution and use the solution file {\tt examples.sln} provided there in the 
\begin{verbatim}
examples\MSVisualStudio
\end{verbatim}
directory.

In order to build the examples in a {\bf unix environment}\index{unix} the user must first run
%
\index{make install@{\tt make install}}
\begin{verbatim}
make install
\end{verbatim}
in the COIN-OS project root directory (the discussion in this section assumes that the project root directory is
{\tt COIN-OS}).  Running {\tt make install}  will  place all the header files required by the examples in the directory
\begin{verbatim}
COIN-OS/include
\end{verbatim}
and all of the libraries required by the examples in the directory
\begin{verbatim}
COIN-OS/lib
\end{verbatim}
In addition the folder {\tt pkgconfig} is placed in the {\tt lib} directory as well. Unix must then be informed of the location of this folder as follows:
\begin{verbatim}
export PKG_CONFIG_PATH=<path to pkgconfig directory>
\end{verbatim}
The source code for the examples is in the {\tt COIN-OS/OS/examples} hierarchy.  For instance, the {\tt osModDemo}
example of section~\ref{section:exampleOSModDemo} is in the directory
\begin{verbatim}
COIN-OS/OS/examples/osModDemo
\end{verbatim}

Next, the user should connect to the appropriate example directory and run {\tt make}.
If the user has done a VPATH\index{VPATH} build, the makefiles\index{makefile|(} will be in each respective example directory under
\begin{verbatim}
vpath_root/OS/examples
\end{verbatim}
otherwise, the makefiles will be in each respective example directory under
\begin{verbatim}
COIN-OS/OS/examples
\end{verbatim}
\else
The binary distribution contains a number of sample applications that illustrate the use of the
OS libraries and other aspects of the OS project. The sample code is found in the {\tt examples}
folder. Each application contains a makefile for unix users; there are also MS Visual Studio project files 
for Windows users. At present only MS Visual Studio 2008 is supported.

Under Windows, connect to the {\tt MSVisualStudio-v9} directory and open {\tt examples.sln} in Visual Studio. All examples can then be built simply by pushing F7 (Build solution). To build only selected examples it is necessary to open the Configuration Manager from the Build menu and select the projects desired to be built.

To build any of the examples under unix, it is at present necessary to set the environment variable
{\tt PKG\_CONFIG\_PATH} to point to the folder {\tt lib/pkgconfig}. Unless some directories were
moved after installing the download, the following unix command will suffice:

\begin{verbatim}
export PKG_CONFIG_PATH=../../lib/pkgconfig
\end{verbatim}

After that, connect to the appropriate directory for the desired project and run {\tt make}. 
For instance, the code and makefile for the {\tt osModDemo}
example of section~\ref{section:exampleOSModDemo} is in the directory
\begin{verbatim}
examples/osModDemo
\end{verbatim}

 
\fi

The {\tt Makefile} in each example directory is fairly simple and is designed to be easily modified 
by the user if necessary.  The part of the Makefile to be adjusted, if necessary, is

\begin{verbatim}
##########################################################################
#    You can modify this example makefile to fit for your own program.   #
#    Usually, you only need to change the five CHANGEME entries below.   #
##########################################################################

# CHANGEME: This should be the name of your executable
EXE = OSModDemo
# CHANGEME: Here is the name of all object files corresponding to the source
#           code that you wrote in order to define the problem statement
OBJS =  OSModDemo.o
# CHANGEME: Additional libraries
ADDLIBS =
# CHANGEME: Additional flags for compilation (e.g., include flags)
ADDINCFLAGS =  -I${prefix}/include
# CHANGEME: SRCDIR is the path to the source code; VPATH is the path to
# the executable. It is assumed that the lib directory is in prefix/lib
# and the header files are in prefix/include
SRCDIR = /Users/kmartin/Documents/files/code/cpp/OScpp/COIN-OS/OS/examples/osModDemo
VPATH = /Users/kmartin/Documents/files/code/cpp/OScpp/COIN-OS/OS/examples/osModDemo
prefix = /Users/kmartin/Documents/files/code/cpp/OScpp/vpath
\end{verbatim}


Developers can use the Makefiles as a starting point for building applications that use the 
OS project libraries\index{makefile|)}.




\subsection{Algorithmic Differentiation:  Using the OS Algorithmic Differentiation Methods}\label{section:cppad}

\index{Algorithmic differentiation|(}
In the {\tt OS/examples/algorithmicDiff} folder is test code {\tt OSAlgorithmicDiffTest.cpp}. This code
illustrates the key methods in the {\tt OSInstance}\index{OSInstance@{\tt OSInstance}} API that are used for
algorithmic differentiation.   These methods are described in Section~\ref{section:ad}.



\subsection{Instance Generator: Using the OSInstance API to Generate Instances}\label{section:exampleOSInstanceGeneration}

This example is found in the {\tt instanceGenerator} folder in the {\tt examples} folder. This example illustrates
how to build a complete in-memory model instance using the {\tt OSInstance}\index{OSInstance@{\tt OSInstance}} API.
See the code {\tt OSInstanceGenerator.cpp} for the complete example. Here we provide a few highlights to illustrate
the power of the API.

The first step is to create an {\tt OSInstance} object.
\begin{verbatim}
OSInstance *osinstance;
osinstance = new OSInstance();
\end{verbatim}

The instance has two variables, $x_{0}$ and $x_{1}$. Variable $x_{0}$ is a continuous variable with lower bound of $-100$ and upper bound of $100$. Variable $x_{1}$ is a binary variable. First declare the instance to have two variables.
\begin{verbatim}
osinstance->setVariableNumber( 2);
\end{verbatim}
Next, add each variable. There is an {\tt addVariable} method with the signature
\begin{verbatim}
addVariable(int index, string name, double lowerBound, double upperBound, char type);
\end{verbatim}
Then the calls for these two variables are
\begin{verbatim}
osinstance->addVariable(0, "x0", -100, 100, 'C');
osinstance->addVariable(1, "x1", 0, 1, 'B');
\end{verbatim}
There is also a method {\tt setVariables} for adding more than one variable simultaneously.  The objective function(s) and constraints are added through similar calls.

Nonlinear terms are also easily added.  The following code illustrates how to add a nonlinear term
$x_{0}*x_{1}$ in the {\tt <nonlinearExpressions>} section of  OSiL. This term is part of constraint~1
and is the second of six constraints contained in the instance.
\begin{verbatim}
osinstance->instanceData->nonlinearExpressions->numberOfNonlinearExpressions = 6;
osinstance->instanceData->nonlinearExpressions->nl = new Nl*[ 6 ];
osinstance->instanceData->nonlinearExpressions->nl[ 1] = new Nl();
osinstance->instanceData->nonlinearExpressions->nl[ 1]->idx = 1;
osinstance->instanceData->nonlinearExpressions->nl[ 1]->osExpressionTree =
new OSExpressionTree();
// the nonlinear expression is stored as a vector of nodes in postfix format
// create a variable nl node for x0
nlNodeVariablePoint = new OSnLNodeVariable();
nlNodeVariablePoint->idx=0;
nlNodeVec.push_back( nlNodeVariablePoint);
// create the nl node for x1
nlNodeVariablePoint = new OSnLNodeVariable();
nlNodeVariablePoint->idx=1;
nlNodeVec.push_back( nlNodeVariablePoint);
// create the nl node for *
nlNodePoint = new OSnLNodeTimes();
nlNodeVec.push_back( nlNodePoint);
// now the expression tree
osinstance->instanceData->nonlinearExpressions->nl[ 1]->osExpressionTree->m_treeRoot =
nlNodeVec[ 0]->createExpressionTreeFromPostfix( nlNodeVec);
\end{verbatim}
\index{Algorithmic differentiation|)}

%\subsection{Excel:  Using VBA To Generate OSiL}\label{section:exampleExcel}

%\subsection{Matlab:  Using  MATLAB To Generate OSiL}\label{section:exampleMatlab}

\subsection{branchCutPrice:  Using Bcp}\label{section:examplebranchCutPrice}

This example illustrates the use of the COIN-OR Bcp (Branch-cut-and-price) project.  
This project offers the user with the ability to have control over each node in the branch and process. 
This makes it possible to add user-defined cuts and/or user-defined variables. At each node in the tree, 
a call is made to the method {\tt process\_lp\_result()}. In the example problem we illustrate 1) adding COIN-OR Cgl cuts, 
2) a user-defined cut, and 3) a user-defined variable. 


\subsection{OSModificationDemo: Modifying an In-Memory {\tt OSInstance} Object}\label{section:exampleOSModDemo}

The {\tt osModificationDemo} folder holds the file {\tt OSModificationDemo.cpp}.
This is similar to the {\tt instanceGenerator} example. In this case, a simple
linear program is generated. However, this example also illustrates how to
modify an in-memory OSInstance object. In particular, we illustrate how to
modify an objective function coeffient. Note the dual occurrence of the
following code

\begin{verbatim}
solver->osinstance->bObjectivesModified = true;
\end{verbatim}

in the {\tt OSModificationDemo.cpp} file (lines 177 and 187).
This line is critical, since otherwise changes made to the OSInstance object
will not be passed to the solver.

This example also illustrates calling a COIN-OR solver,
in this case {\tt Clp}\index{COIN-OR projects!Clp@{\tt Clp}}.

\vskip 8pt

{\bf Important:} the ability to modify a problem instance is still extremely limited in this release.
A better API for problem modification will come with a later release of OS.



\subsection{OSSolverDemo: Building In-Memory Solver and Option Objects}\label{section:exampleOSSolverDemo}

The code in the  example file {\tt OSSolverDemo.cpp} in the folder {\tt osSolverDemo}  illustrates  how to build solver interfaces and  an in-memory {\tt OSOption} object. In this example we  illustrate building a solver interface and corresponding {\tt OSOption} object for the solvers {\tt Clp}, {\tt Cbc}, {\tt SYMPHONY}, {\tt Ipopt},   {\tt Bonmin}, and {\tt Couenne}.   Each solver class inherits from a virtual {\tt OSDefaultSolver} class. Each solver class has the string data members

\begin{itemize}
\item {\tt osil --} this string conforms to the OSiL standard and holds the model instance.

\item {\tt osol --} this string conforms to the OSoL standard and holds an instance with the 
solver options (if there are any); this string can be empty.

\item {\tt osrl --} this string conforms to the OSrL standard and holds the solution instance; 
each solver interface produces an osrl string.
\end{itemize}

Corresponding to each string there is an in-memory object data member, namely

\begin{itemize}
\item {\tt osinstance --}  an in-memory {\tt OSInstance} object containing the model instance
and get() and set() methods to access various parts of the model.


\item {\tt osoption --} an in-memory {\tt OSOption} object; solver options can be accessed or 
set using get() and set() methods.


\item {\tt osresult --}  an in-memory {\tt OSResult} object; various parts of the model solution  
are accessible through get() and set() methods.
\end{itemize}


For each solver we detail five steps:

\begin{itemize}
\item[Step 1:]  Read a model instance from a file  and create the corresponding {\tt OSInstance} object.
For four of the solvers we read a file with the model instance in OSiL format. For the Clp example 
we read an MPS file and convert to OSiL. For the Couenne example we read an AMPL nl file and convert 
to OSiL.

\item[Step 2:]  Create an {\tt OSOption} object and set options appropriate for the given solver.   
This is done by defining

\begin{verbatim}
OSOption* osoption = NULL;
osoption = new OSOption();
\end{verbatim}

A key method in the {\tt OSOption} interface is {\tt setAnotherSolverOption()}.  This method 
takes the following arguments in order.

\begin{itemize}
\item[] {\tt std::string name} -- the option name;
\item[] {\tt std::string value}  -- the value of the option;
\item[] {\tt std::string solver} -- the name of the solver to which the option applies;
\item[] {\tt std::string category} -- options may fall into categories. For example, consider the  
Couenne solver.  This solver is also linked to the Ipopt and Bonmin solvers and  it is possible 
to set options for these solvers through the Couenne API. In order to set an Ipopt option 
you would set the {\tt solver} argument to {\tt couenne} and set the {\tt category} option 
to {\tt ipopt}.

\item[] {\tt std::string type} -- many solvers require knowledge of the data type, so you can set 
the type to {\tt double}, {\tt integer}, {\tt boolean} or {\tt string}, depending on the solver 
requirements. Special types defined by the solver, such as the type {\tt numeric} used by the
Ipopt solver, can also be accommodated. It is the user's responsibility to verify the type
expected by the solver.


\item[] {\tt std::string  description} -- this argument is used to provide any detail or 
additional information about the option. An empty string ({\tt""}) can be passed if such additional
information is not needed.
\end{itemize}

For excellent documentation that details solver options for Bonmin, Cbc, and Ipopt  we recommend 

\begin{center}
\url{http://www.coin-or.org/GAMSlinks/gamscoin.pdf}
\end{center}


\item[Step 3:] Create the solver object. In the OS project there is a {\it virtual} solver that 
is declared by

\begin{verbatim}
DefaultSolver *solver  = NULL;
\end{verbatim}

The Cbc, Clp and SYMPHONY solvers as well as other solvers of linear and integer linear programs
are all invoked by creating a {\tt CoinSolver().} For example, the following is used to invoke Cbc.

\begin{verbatim}
solver = new CoinSolver();
solver->sSolverName ="cbc";
\end{verbatim}

%Then to declare a specific, for example, an {\tt Ipopt} solver, simply write
Other solvers, particularly Ipopt, Bonmin and Couenne are implemented separately. So to declare,
for example, an Ipopt solver, one should write

\begin{verbatim}
solver = new IpoptSolver();
\end{verbatim}

The syntax is the same regardless of solver. 

\item[Step 4:] Import the {\tt OSOption} and {\tt OSInstance} into the solver and solve the model. 
This process is identical regardless of which solver is used. The syntax is:

\begin{verbatim}
solver->osinstance = osinstance;
solver->osoption = osoption;	
solver->solve();
\end{verbatim}

\item[Step 5:] After optimizing the instance,  each of the OS solver interfaces uses the underlying solver API to get the solution result and write the result to a string 
named {\tt osrl} which is a string representing the solution instance in the {\tt OSrL} XML standard.  
This string is accessed by

\begin{verbatim}
solver->osrl
\end{verbatim}


In the example code {\tt OSSolverDemo.cpp} we have written a method,  

\begin{verbatim}
void getOSResult(std::string osrl)
\end{verbatim}

that takes the {\tt osrl} string and creates an {\tt OSResult} object.   
We then illustrate several of the {\tt OSResult} API methods 

\begin{verbatim}
double getOptimalObjValue(int objIdx, int solIdx);
std::vector<IndexValuePair*>  getOptimalPrimalVariableValues(int solIdx);
\end{verbatim}
to get and write out the optimal objective function value, and optimal primal values.  See also Section \ref{section:exampleOSResultDemo}.

\end{itemize}

We now highlight some of the features illustrated by each of the solver examples.

\begin{itemize}
\item {\bf Clp --}  In this example we read in a problem instance in MPS format.  The class 
{\tt OSmps2osil}  has a method {\tt mps2osil} that is used to convert the MPS instance contained 
in a file into an in-memory {\tt OSInstance} object. This example also illustrates how to 
set options using the Osi interface. In particular we turn on intermediate output which is 
turned off by default in the Coin Solver Interface. 

\item {\bf Cbc --}  In this example we read a problem instance that is in OSiL format and create 
an in-memory {\tt OSInstance} object.  We then create an {\tt OSOption} object.  This is quite trivial.  
A  plain-text XML file conforming to the OSiL schema is read into a string {\tt osil} which is then 
converted into the in-memory {\tt OSInstance} object by

\begin{verbatim}
OSiLReader *osilreader = NULL;
OSInstance *osinstance = NULL;
osilreader = new OSiLReader(); 
osinstance = osilreader->readOSiL( osil);
\end{verbatim}


 We set the linear programming algorithm to be the primal simplex method and then set the option 
on the pivot selection to be Dantzig rule.  Finally, we set the print level to be 10.

\item {\bf SYMPHONY --}   In this example we also read a problem instance that is in OSiL format and 
create an in-memory {\tt OSInstance} object.  We then create an {\tt OSOption} object and 
illustrate setting the {\tt verbosity} option.

\item {\bf Ipopt --}   In this example we also read a problem instance that is in OSiL format.  
However, in this case we do  not create an {\tt OSInstance} object. We read the OSiL file into 
a string {\tt osil}.  We then feed the {\tt osil} string directly into the Ipopt solver by
\begin{verbatim}
solver->osil = osil;
\end{verbatim} 
The user always has the option of providing the OSiL to the solver as either a string or in-memory object.

Next we create an {\tt OSOption} object. For Ipopt, we illustrate setting the maximum iteration limit 
and also provide the name of the output file. In addition, the OSOption object can hold initial solution 
values. We illustrate how to initialize all of the variable to 1.0.

\begin{verbatim}
numVar = 2; //rosenbrock mod has two variables 
xinitial = new double[numVar];
for(i = 0; i < numVar; i++){
    xinitial[ i] = 1.0;
}
osoption->setInitVarValuesDense(numVar, xinitial);
\end{verbatim}



\item {\bf Bonmin --}  In this example we read a problem instance that is in OSiL format and create 
an in-memory {\tt OSInstance} object just as was done in the Cbc and SYMPHONY examples.   
We then create an {\tt OSOption} object.  In setting the  {\tt OSOption} object we intentionally 
set an option that will cause the Bonmin solver to terminate early.  In particular we set the 
{\tt node\_limit} to zero. 

\begin{verbatim}
osoption->setAnotherSolverOption("node_limit","0","bonmin","","integer","");
\end{verbatim}

This results in early termination of the algorithm. The {\tt OSResult} class API has a method
\begin{verbatim}
std::string getSolutionStatusDescription(int solIdx);
\end{verbatim}

For this example, invoking
\begin{verbatim}
osresult->getSolutionStatusDescription( 0)
\end{verbatim}
gives the result:
\begin{verbatim}
LIMIT_EXCEEDED[BONMIN]: A resource limit was exceeded, we provide the current solution.
\end{verbatim}


\item {\bf Couenne --}   In this example we read in a problem instance in AMPL nl format.  
The class {\tt OSnl2osil}  has a method {\tt nl2osil} that is used to convert the nl instance 
contained in a file into an in-memory {\tt OSInstance} object. This is done as follows:

\begin{verbatim}
// convert to the OS native format
OSnl2osil *nl2osil = NULL;
nl2osil = new OSnl2osil( nlFileName);
// create the first in-memory OSInstance
nl2osil->createOSInstance() ;
osinstance =  nl2osil->osinstance;
\end{verbatim}
\end{itemize}

This part of the example also illustrates setting options in one solver from another. 
Couenne uses Bonmin which uses Ipopt.  So for example,

\begin{verbatim}
osoption->setAnotherSolverOption("max_iter","100","couenne","ipopt","integer","");
\end{verbatim}
identifies the solver as {\tt couenne}, but the category of value of {\tt ipopt}  tells the solver 
interface to set the iteration limit on the Ipopt algorithm that is solving the continuous relaxation 
of the problem.  Likewise, the setting
\begin{verbatim}
osoption->setAnotherSolverOption("num_resolve_at_node","3","couenne","bonmin","integer","");
\end{verbatim}
identifies the solver as {\tt couenne}, but the category of value of {\tt bonmin}  tells the solver 
interface to tell the Bonmin solver to try three starting points at each node. 

 

\subsection{OSResultDemo: Building In-Memory Result Object to Display Solver Result}\label{section:exampleOSResultDemo}

The OS protocol for representing an optimization result is {\tt OSrL}. Like the {\tt OSiL} and {\tt OSoL} protocol, this protocol has an associated in-memory {\tt OSResult} class with corresponding API.  The use of the API is demonstrated in the code {\tt OSResultDemo.cpp} in the folder {\tt OS/examples/OSResultDemo}.  In the code we solve a linear program with the {\tt Clp} solver.  The OS solver interface builds an {\tt OSrL} string that we read into the {\tt OSrLReader} class and create and {\tt OSResult} object. We then use the {\tt OSResult} API to get the optimal primal and dual solution. We also use the API to get the reduced cost values. 


\subsection{OSCglCuts: Using the OSInstance API to Generate Cutting Planes}\label{section:exampleOSAddCuts}

In this example, we show how to add cuts to tighten an LP using COIN-OR
{\tt Cgl} (Cut Generation Library)\index{COIN-OR projects!Cgl@{\tt Cgl}}.
A file ({\tt p0033.osil}) in OSiL format is used to create an OSInstance object. The linear programming relaxation
is solved. Then, Gomory, simple rounding, and knapsack cuts are added using {\tt Cgl}.  The model is then optimized
using {\tt Cbc}.

\subsection{OSRemoteTest:  Calling a Remote Server}\label{section:exampleOSRemoteTest}

This example illustrates the API for the six service methods described in Section~\ref{section:servicemethods}.
The file {\tt osRemoteTest.cpp} in folder {\tt osRemoteTest} first builds a small linear
example, solves it remotely in synchronous mode and displays the solution.
The asynchronous mode is also tested by submitting the problem to a remote solver,
checking the status and either retrieving the answer or killing the process if it has not
yet finished.

{\bf Windows users should note}
that this project links to {\tt wsock32.lib}, which is not part of the Visual Studio  Express Package.  It is necessary
to also download and install the Windows Platform SDK\index{Windows Platform SDK}, which can be found at

\medskip
\noindent{\scriptsize\tt\UrlSdk}. 
\medskip

\ifdevelop
\noindent See also Section~\ref{section:msvs}.
\else
\noindent Further information is provided in the OS User's Manual.
\fi

\subsection{OSJavaInstanceDemo:  Building an OSiL Instance in
Java}\label{section:exampleOSJavaDemo}
\index{Java|(}

In this example we demonstrate how to build an OSiL instance using the Java
OSInstance API.  The example code also  illustrates calling the {\tt
OSSolverService} executable from Java. In order to use this example, the user should do an svn
checkout:

\begin{verbatim}
svn co https://projects.coin-or.org/svn/OS/branches/OSjava OSjava
\end{verbatim}

The {\tt OSjava} folder contains the file {\tt INSTALL.txt}. Please follow the
instructions in {\tt  INSTALL.txt} under the heading:
\begin{verbatim}
== Install Without a Web Server==
\end{verbatim}

These instructions assume that the user has installed the Eclipse IDE. See
\url{http://www.eclipse.org/downloads/}. At this link we recommend that the 
user get {\tt Eclipse Classic}.  In addition, the user should also have a copy of the
{\tt OSSolverService} executable that is compatible with his or her platform.
The {\tt OSSolverService} executable for several different platforms is
available at \url{http://www.coin-or.org/download/binary/OS/OSSolverService/}. 
The user can also build the executable as described in this Manual.  See Section
\ref{section:build}. The code base for this example is in the folder:
\begin{verbatim}
OSjava/OSJavaExamples/src/OSJavaInstanceDemo.java
\end{verbatim}
The code in the file {\tt OSJavaInstanceDemo.java} demonstrates how the
Java OSInstance API that is in {\tt OSCommon} can be used to generate a linear
program and then call the C++ {\tt OSSolverService} executable 
to solve the problem.\index{Java|)}  Running this example in Eclipse will
generate in the folder
\begin{verbatim}
OSjava/OSJavaExamples
\end{verbatim}
two files. It will generate {\tt parincLinear.osil} which is a linear program in
the OS OSiL format, it will also call the {\tt OSSolverService} executable which
generates the result file {\tt result.osrl} in the OS OSrL format. 


